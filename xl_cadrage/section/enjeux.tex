
La Prefecture dispose d'un grand nombre de documents qui nécessitent d'être numérisés et classés par un système indépendant de l'intervention humaine, source d'erreur.
Ce travail laborieux est actuellement réalisé par des employés, ce qui rend cette tâche couteuse en temps et en ressources humaines.

C'est la la source d'erreur principal de ce classement : chaque employés utilisent des systèmes de classement différents, et n'emploient pas les mêmes pratiques d'archivage.
Cette différence dans les méthodes de classement se révèle être un problème lors de la recherche des documents archivés par la suite.
Ces recherches peuvent alors prendre beaucoup de temps, voir même ne pas aboutir.
\\
De plus certains documents peuvent se retrouver dupliqués, périmés ou incomplets, ce qui rend leur utilité moindre.
\\
\par
Ce projet permettra un gain de temps sur l'archivage et les recherches documentaires, et permettra une accélération durable des services préfectoraux.


