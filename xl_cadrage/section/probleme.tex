%Quel est le probleme
%Qui a le probleme
%Comment le probleme se présente t'il
%En quoi est-ce un probleme
%Quels sont les enjeux ?
La préfecture de la Mayenne doit, par obligation légale, conserver la totalité des documents administratifs produits et traités.
La durée de conservation dépend du type de document; cela peut aller a quelques années pour des documents mineurs, à une durée indéfinie.
En faisant ainsi, l'administration garantie une traçabilité de ses actions et permet une certaine transparence. 
\\
\par
Cependant, la production documentaire de l'administration est considérable, et cela d'autant plus avec l'arrivée de l'informatique qui permet de créer et valider des documents bien plus rapidement qu'auparavant.
Chaque jour des dizaines de nouveaux documents (arrêtés, formulaires, ...) sont crées et doivent être archivés pour leur conservation.
\\
\par
A ce jour, il n'y a pas, ou très peu de règles permettant une uniformisation de l'organisation de ces documents.
Bien qu'il existe une taxonomie officielle, qui permet de définir très précisément le type d'un document et son contenu, celle ci est bien trop lourde et complexe (plus de 6000 catégories) pour qu'elle soit utilisée naturellement.
Chaque employés administratif utilise donc sa propre méthode organisationnelle, et organise les centaines de documents qu'il doit traiter de la manière qui lui semble la plus logique/intuitive.
Ce manque d'uniformité et de rationalité dans l'organisation contribue a un ralentissement du processus administratif, qui nécessite par sa nature une grande quantité de document spécifiques et très précis qui ne peuvent pas être trouvés facilement dans la masse.
Pour la préfecture de la Mayenne, on aurait affaire à environ 80 téraoctets de documents divers et variés, ce qui est considérable pour des documents écrits. 
\\
\par
Ce problème est ressenti par la préfecture de la Mayenne, mais est aussi retrouvé chez toute organisation d'envergure qui produit une grande quantité de document. 
\\
\par
Il est important de préciser que si le document en particulier répond a une organisation précise, et est bien défini \textit{en lui même}, il ne l'est pas dans \textit{l'ensemble}.
En somme, si nous pouvons définir très précisément quel type de document il s'agit, il est difficile de le retrouver dans la masse de documents produits.
Il n'y a quasiment jamais de métadonnées (taxonomie, mots clefs, ...) attachées à celui-ci, et il peut être caché dans une dense forêt de répertoires a la nomenclature différente pour chaque employés administratif.
\\
\par
De plus, les documents administratifs possèdent des versions mineures et majeures.
Les versions mineures sont des versions de travail, qui peuvent contenir des fautes et sont destinées a être peaufinées pour devenir des documents majeurs, des copies propres qui peuvent êtres officiellement validées.
Il arrive parfois que les versions mineures de documents soient conservées, ce qui complique encore plus la tâche.
\emph{La Prefecture estime que la plupart de son espace occupé correspond à des versions mineures ou des copies de documents.}
\\
\par
Il s'agit ici de rationaliser l'organisation documentaire de la préfecture, et de permettre de rendre plus efficace le processus administratif qui est considérablement ralenti par la recherche constante des documents.
Ce ralentissement a bien évidement un coût monétaire important, et est très frustrant pour les employés administratifs comme pour les citoyens ayant affaire a l'administration, que ce soit en Mayenne ou en France de façon générale. 


