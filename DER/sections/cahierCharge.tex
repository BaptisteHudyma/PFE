%Description des fonctions à servir (diagramme des cas d'utilisation)

\subsection {Cadre de travail}
La quantité de documents non organisé de la prefecture de Mayenne est considérable.
Aussi, réaliser un système permettant de traiter tous ces documents, de styles et de types très différents, n'était pas réalisable dans les 6 mois du projet.
Nous avons et le commanditaire on donc convenu de la réalisation d'un \textit{Proof Of Concept} se focalisant uniquement sur les documents de type Recueil des Actes Administratifs (RAA).
Ces documents sont publiés publiquement au moins une fois par mois, et contiennent tous les actes a caractère réglementaires non personnels publiés durant le mois dernier.
Certains RAA qualifiés de \textit{spéciaux} peuvent également être publiés a titre exceptionnel durant le mois.
On observe en moyenne 130 RAA publiés par ans.

Dans le cadre du projet, nous avons récupéré tous les RAA de l'année 2016 à 2019, c'est a dire 431 documents.
Chaque RAA regroupe plusieurs arrêtés pouvant traiter de sujets très diverses, de l'interdiction de manifestations à la délégation de signature.

Le cadre d'étude du projet est donc très large mais suffisamment restreinte pour qu'il soit possible de produire un POC dans les délais impartis. 

\subsection{Recherche d'informations d'importance} %Baptiste
Chaque document contient des informations d'importance qui doivent être extraites pour que le document en question soit correctement classé par la suite.
Les informations d'importance, déterminées avec le commanditaire, sont les suivantes :
\begin {itemize}
\item Référence du RAA
\item Nom du signataire
\item Date de publication
\item Dates mentionnées
\item Références des arrêtés
\item Références des décrets
\item Références des articles
\item Références des lois
\item Noms des parties prenantes
\item Lieux mentionnés
\item Noms d'organisation et d'organismes
\end {itemize}



\subsection{Classement par taxonomie}
Pour pouvoir obtenir une classification précise des documents administratifs, une taxonomie contenant plus de 6000 termes a été développée par l'administration Française. Cette taxonomie permet de tagger précisément le contenu d'un document.
Il était donc impératif de pouvoir extraire depuis le texte assez d'information pour pouvoir assigner au document une ou plusieurs taxonomies.

\subsection{Moteur de recherche}

