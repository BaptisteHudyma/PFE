\newacronym{poc}{PoC}{Proof of Concept}
\newacronym{raa}{RAA}{Recueil des Actes Administratifs}
\newacronym{nlp}{NLP}{Natural Language Processing}
\newacronym{ui}{UI}{User Interface}
\newacronym{ocr}{OCR}{Optical Character Recognition}
\newacronym{cnn}{CNN}{Convolutional Neural Network}
\newacronym{crnn}{CRNN}{Convolutional Recurrent Neural Network}
\newacronym{gcc-phat}{GCC-PHAT}{Generalized Cross Correlation with Phase Transform}
\newacronym{lstm}{LSTM}{Long Short Term Memory}
\newacronym{rnn}{RNN}{Recurent Neural Network}
\newacronym{dnn}{DNN}{Deep Neural Network}
\newacronym{json}{JSON}{JSON}


\newglossaryentry{JSON}{
	name=JSON,
	description={JavaScript Object Notation est un format de données textuelles dérivé de la notation des objets du langage JavaScript.}
}

\newglossaryentry{OCR}{
	name=OCR,
	description={L' Optical Character Recognition désigne les procédés informatiques pour la traduction d'images de textes imprimés ou dactylographiés en fichiers de texte.}
}

\newglossaryentry{NLP}{
	name=NLP,
	description={Le Natural Langage Processing est un domaine multidisciplinaire impliquant la linguistique, l'informatique et l'intelligence artificielle, qui vise à créer des outils de traitement de la langue naturelle pour diverses applications}
}

\newglossaryentry{CNN}{
	name=CNN,
	description={Réseau de convolution. Type de réseau neuronal, qui apprends des poids dans des matrices, qui seront convoluées, et produiront une \textit{feature-map}. En général, on aura une d'autres couches de convolution, qui appliquera d'autres filtres sur la \textit{feature-map} ainsi obtenue}
}


\newglossaryentry{finetune}{
	name=finetune,
	description={Le finetuning consiste à effectuer de très légères modifications dans le réseau afin d'obtenir de meilleurs résultats. On modifiera par exemple la taille des batchs lors de l'apprentissage, la taille des filtres, où leur strides. On le pratique après avoir entrainé une première version du réseau qui sert de prototype et donne une idée des résultats que l'approche peut obtenir}
}

\newglossaryentry{ErrorRate}{
	name=Error Rate,
	description={Voir définition (\ref{ErrorRate})}
}

\newglossaryentry{deeplearning}{
	name=Deep Learning,
	description={Ensemble de techniques utilisant des réseaux de neurones pour effectuer des taches de Machine Learning. On entraine ces réseaux a effectuer ces taches en leur présentant des grandes quantités d'exemples et leur réponses, et on modifie leurs paramètres (parfois au nombres de plusieurs millions) pour réduire une fonction de perte. Cette fonction de perte indique a quel point le réseau effectue correctemement la tache demandé.}
} 

\newglossaryentry{embed}{
	name=embeddings,
	description={Vecteur de taille fixe représentant un mot dans un espace vectoriel. Ce vecteur est genéré par un réseau de neurones peu profond, qui apprends a construire cette représentation en fonction du contexte dans lequel se trouve le mot. Des mots partageant un contexte similaire auront donc des vecteurs ayant une distance faible.}
} 
