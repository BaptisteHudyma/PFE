
\section{Prise de recul}
Notre équipe a unanimement apprécié ce projet.
Le challenge proposé par un aussi gros projet en si peu de temps était stimulante pour l'esprit aussi bien personnel que de groupe.
Cela nous a poussé a favoriser grandement le travail d'équipe.

Ce PFE ambitieux nous a renforcé dans notre posture d’ingénieur. 
Nous sommes satisfait du résultat final et nous sommes persuadés que nous avons pu atteindre cette finalité grâce à notre motivation, notre complémentarité et la passion de chacun. 

\subsection{Enseignements}
L'utilisation des méthodes agiles dans un contexte réel nous a donné un aperçu de l'efficacité qu'on peut tirer de leurs utilisation.
Un outil comme Trello a changé la donne pour le projet: Il était soudain devenu facile de savoir quel était l'état de chaque tâche et d'en prévoir le futur, de prévenir des retards.
Grâce a ces techniques organisationnelles, nous avons pu finir le POC dans le temps imparti.
Notre commanditaire est d'ailleurs très satisfait du travail fourni, ce qui justifie également le succès des méthodes de travail employées.

La répartition intelligente du travail en tâches principales sous la responsabilité personnelle d'une personne a la fois nous a permis de chacun connaitre parfaitement notre tâche.
Cela nous a permis de chacun nous comporter comme des manager auprès des autres membres, en étant responsable de la division de notre tâche principale en sous tâches, puis la distribution des sous tâches aux autres membres de l'équipe.
Le travail de groupe était donc le centre du projet, et cette expérience de travail était agréable.
Nous avons également améliorés nos compétences sur des outils tel que Git et \LaTeX qui nous permettent de travailler de manière efficace et collaborative.

Utiliser des PowerPoint pour les réunions avec le mentor nous a également facilité beaucoup de temps en explications.
Présenter des diagrammes Gantt pour montrer l'état du projet est bien plus évocateur que de juste décrire la situation.
De même, des schémas techniques valent milles explications.
L'utilisation des supports visuels nous a surtout été démontré grâce a Staberlin, très doué dans ce domaine.

Nous avons également beaucoup appris sur certains systèmes administratifs, notamment le fonctionnement des RAA, des arrêtés et des archives.
Travailler pour un domaine aussi complexe sur des problèmes humains (la classification) nous a permis découvrir une nouvelle vision du travail prefectoral.
De même, nous avons découvert la tâche très compliquée qu'est la classification.
Sa complexité en fait un métier a part entière.

Nous allons bientôt commencer notre stage de fin d'étude, et les connaissances techniques et managériales que nous avons développé nous seront certainement utiles.

\subsection{Points forts}
Notre équipe était extrêmement bien équilibrée.
Les faiblesses/lacunes des uns étaient parfaitement compensées par les forces des autres, et cela nous a permis d'être très efficaces.

De plus nous avons entretenus un esprit de camaraderie dans le travail.
La bonne ambiance de travail et le fonctionnement de l'équipe sont certainement des facteurs majeurs de la réussite du projet.

Notre équipe est composée de talents divers qui ont contribué chacun a leur façon a rendre le projet viable.
Nous considérons aussi que nous possédons de bonnes compétences techniques et managériales.

Nous avons également eu une très bonne relation avec notre commanditaire, qui a toujours su se rendre disponible.
Nous avons pu faire des réunions assez souvent, ce qui nous a permis de réaliser un POC qui correspond a ses attentes.
Il nous a également organisé une visite des archives départementale afin que nous puissions comprendre un peu mieux le problème technique qu'est la classification.


\subsection{Axes d'amélioration}
Notre équipe, bien que très équilibrée, n'est pas parfaite.
Nous avons une tendance a sous estimer les tâches, qu'elles soient techniques ou pas.
La rédaction du DER a par exemple pris plus de 2 fois le temps de rédaction prévu, ce qui nous a conduit a le finir pendant les jours ou les autres tâches avançaient mieux, au détriments d'autres tâches qui auraient en rétrospective mérités plus d'attention.

Nous avons également plusieurs fois mal utilisé Trello, en n'y mentionnant pas des tâches d'importance comme la rédaction des rapports de réunion, ou la préparation des réunions.
Cela nous a fait occasionnellement prendre du retard sur d'autres tâches pour rattraper le temps, ou oublier simplement de rédiger des rapports de réunion.

Techniquement notre POC est fonctionnel mais il y a encore beaucoup d'améliorations a y apporter pour qu'il soit vraiment intéressant.
Actuellement, nous n'avons pas mis de système de correction des erreurs de transcript sur les textes en sortie de l'OCR.
Cela fait les transcripts des documents examinés par OCR contiennent encore beaucoup d'erreurs de caractères, comme des `e' transformés en `c'.

Notre approche de récupération de données est également améliorable car elle n'utilise pour le moment que des règles `dure'.
Les cas gérés sont donc limités a ce que nous avons observé sur un petit échantillon de documents.
Ce problème existe également pour la taxonomie.
Notre méthode de récupération de la taxonomie est très approximative (bien que fonctionnelle).
Une méthode plus efficace mettrait en jeux toute la chaine taxonomique qui conduit au terme pour en étudier le contexte réel, mais cela aurait nécessité une base de donnée.

Notre moteur de recherche gagnerait également a montrer des informations utiles sur les documents affichés, comme une page internet dédiée contenant les titres des arrêtés mentionnés dedans.
De même, un lien vers le transcript du document n'est pas forcément utile: un lien vers le document lui même est bien plus intéressant.
Les RAA étant stockés en ligne, nous pensé a mettre les liens vers les documents réels, mais cette tâche était trop secondaire et nous n'avons pas eu le temps de la réaliser.










