%Plan et documentation de déploiement

%déploiement sur appbase.io

Le déploiement de l'application se passe en trois étapes.

Premièrement, les documents à traiter (au format pdf) doivent être placés dans un dossier, où ils seront récupérés un par un par un script qui va en extraire le texte brut, qui sera lui même analysé pour en obtenir les données.

Ensuite, ces données sont ajoutées à un fichier JSON dont le format correspond à ce que Reactivesearch attend.

Enfin le JSON ainsi crée doit être uploadé dans notre moteur de recherche.

\subsection{Récupération du transcript d'un document}
La chaine de traitement commence par l'extraction du texte des documents PDF.
Pour cela, le script `extractText.sh' va tester tous les documents selon la méthode suivante:

Utilisation de pdftotext: Si du texte peut être extrait du PDF courant, ce texte est stocké dans un nouveau fichier et le script passe au document suivant.

Si aucun texte n'a pu être extrait, le script charge Tesseract (\gls{OCR}) en mémoire, découpe le document en pages stockées au format TIF qui sont individuellement passées par un script de traitement d'image.
Les images traitées individuellement sont alors rassemblées dans une image au format TIFF, transmise à Tesseract qui se charge d'en extraire le texte.

L'extraction dure environ 1h10 pour nos 431 fichiers.

Quand l'exécution de ce script est terminée, on obtient un dossier contenant les transcripts de chaque fichier traités en format TXT\@.

\subsection{Extraction des données}
L'utilisation du script python `reactiv\_json.py' permet de créer un fichier au format JSON pouvant être lu par Reactivesearch.
Ce script commence par lire chaque fichier de transcript au format TXT crées précédemment et en extraire les informations avec les bibliothèques Python situées dans le dossier `utils'.

L'extraction de toutes les données pour le format JSON dure environ 40 minutes pour nos 431 documents.

Le fichier crée par ce script est directement utilisable avec le moteur de recherche.

Un autre script appelé `elastic\_json.py' permet de créer un document JSON dont la syntaxe correspond à ElasticSearch.

\subsection{Moteur de recherche}
%Le moteur de recherche fonctionne comme machin machin machin
Le moteur de recherche basé sur ElasticSearch et Reactivesearch est utilisable tel quel grâce à \href{https://ujcqr.csb.app}{ce lien}.
Il est directement disponible en ligne et ne nécessite pas de nouvelle installation ou de logiciel supplémentaire.

Cependant le moteur de recherche n'est pas interactif: aucune action n'est directement possible sur les données.
Pour modifier, ajouter, retirer des données au site internet, il faut directement passer par l'application Appbase.io ou le JSON contenant nos données sont stockées.

L'import/modification/suppression des données peut être réaliser en ligne.
L'import d'un nouveau JSON est également possible si le format du JSON respecte certaines normes. 

Lorsque les données sont correctement importées, un mail contenant l'identifiant du JSON (credentials) afin de spécifié l'utilisation des données. 

Cette méthode n'est pas très ergonomique mais l'ergonomie de cette partie n'est pas nécessaire pour ce \gls{poc}\@.
