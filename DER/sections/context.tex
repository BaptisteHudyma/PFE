
\subsection{Contexte}

Ce projet est proposé par la Prefecture de la Mayenne dans un contexte de dématérialisation des documents administratifs.
En effet dans un effort de numérisation de leur base documentaire, il devient complexe de gérer la grande quantité/diversité de documents informatisés. 
De plus, le facteur humain de la classification non automatisée n'est pas une garantie d'archivage pérenne.
En effet les employés administratifs changent, et d'une personne à l'autre la stratégie d'archivage peut et va varier.


C'est pourquoi la Prefecture de Mayenne envisage un projet de classification automatisée.
La mise en oeuvre de ce projet permettra de faciliter la recherche de documents, accélérer les démarches et réduire la masse documentaire de l'État (quantité de documents stockés), tout en garantissant que le système d'archivage garde une logique constante, indépendante du facteur humain.



\subsection{Analyse du problème}
%Plan : 
%1) la numérisation et le classement prend du temps
%2) c'est source d'erreurs car les personnes ne classent pas toutes de la même façon
%3) comme le classement n'est pas toujours correct, les recherches peuvent prendre du temps voire ne pas aboutir
%4) peut-être y-a-t-il des doublons???
%5) des documents périmés?....
%6) des documents incomplets ??? (toutes les pages pas présentes? ou bien les PJ rattachées incomplètes?)

%gain de temps des recherches
%securisation de l'information ?
%vérification de la complétude des dossiers


La Prefecture dispose d'un grand nombre de documents qui nécessitent d'être numérisés et classés par un système indépendant de l'intervention humaine, source d'erreur.
Ce travail laborieux est actuellement réalisé par des employés, ce qui rend cette tâche couteuse en temps et en ressources humaines.

C'est la la source d'erreur principal de ce classement : chaque employés utilisent des systèmes de classement différents, et n'emploient pas les mêmes pratiques d'archivage.
Cette différence dans les méthodes de classement se révèle être un problème lors de la recherche des documents archivés par la suite.
Ces recherches peuvent alors prendre beaucoup de temps, voir même ne pas aboutir.

De plus certains documents peuvent se retrouver dupliqués, périmés ou incomplets, ce qui rend leur utilité moindre.

Ce projet permettra un gain de temps sur l'archivage et les recherches documentaires, et permettra une accélération durable des services préfectoraux.


\subsection{Périmètre}
Le type de documents a analyser ont été définis précisément lors de la réunion du vendredi 18 octobre 2019 sur site avec le commanditaire.

La quantité de documents non organisés de la prefecture de Mayenne est considérable.
Aussi, réaliser un système permettant de traiter tous ces documents, de styles et de types très différents, n'était pas réalisable dans les 6 mois alloués au projet.
Nous et le commanditaire avons donc convenu de la réalisation d'un \textit{Proof Of Concept} se focalisant uniquement sur les documents de type `Recueil des Actes Administratifs' (RAA).
Ces documents sont publiés publiquement au moins une fois par mois, et contiennent tous les actes à caractère réglementaire non personnels publiés durant le mois dernier.
Certains RAA qualifiés de \textit{spéciaux} peuvent également être publiés à titre exceptionnel durant le mois.
On observe en moyenne 130 RAA publiés par an.

Dans le cadre du projet, nous avons récupéré tous les RAA de l'année 2016 à 2019, c'est a dire 431 documents.
Chaque RAA regroupe plusieurs arrêtés pouvant traiter de sujets très divers, de l'interdiction de manifestation à la délégation de signature.

Le cadre d'étude du projet est donc très large mais suffisamment restreint pour qu'il soit possible de produire un POC dans les délais impartis. 
