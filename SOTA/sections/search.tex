\subsection{Elastic Search}
Créé en 2004, Elastic search est un moteur de recherche open source basé sur le logiciel libre Apache Lucene. C’est une solution construite pour être distribuée et pour utiliser du JSON via des requêtes HTTP, ce qui rend le moteur de recherche utilisable avec n'importe quel langage de programmation. Il se base sur l’indexation des documents orienté texte. Après une demande de recherche, cette approche lui permet de ne pas avoir à examiner l’ensemble des documents sur la base de données. 
Pour fonctionner, Elasticsearch a besoin de savoir quels mots sont employés dans chaque document. Pour cela, il intègre un moteur Lucene qui va s’occuper d’extraire les mots d’une collection de documents et de préparer des colonnes de mots.
La pertinence de la recherche est donnée par le calcul d’un score par document : plus le document ressemble à la requête, plus il aura un score élevé. 
