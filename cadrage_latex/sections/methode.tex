
\subsection {Méthode de gestion du projet}

Au sein de l'équipe, nous avons adoptés une méthode de fonctionnement agile avec une organisation du projet autour d'un plan IVVQ.
Le plan IVVQ a été validée par le client, et résume les caratéristiques principales du projet ainsi que les tests techniques et fonctionnels qui seront réalisés pour assurer le fonctionnement final du projet.


Nous avons décidé de séparer le projet en trois blocs principaux.
Chaque membre de l'équipe est responsable d'un bloc du projet, c'est à dire de la façon dont il va être réalisé.
Toutes fonction d'un bloc principal peut être confiée à un autre membre du groupe qui est plus à l'aise avec l'approche technique du problème.


La rédaction des documents de rendu comme celui ci sera réalisée avec le langage \LaTeX.


L'intégralité du code et des documents \LaTeX sera géré sur GitHub, à l'adresse \url{https://github.com/smallito/PFE}.
Les documents \LaTeX compilés seront ensuite placés sur le sharepoint.



\subsection {Réalisation Technique}

Nous avons séparé le projet en plusieurs parties principales :
\begin {itemize}
\item \textbf {Reconnaissance des caractères} :
Le document doit être analysé afin d'en retirer le contenu textuel.
\\
\item \textbf {Analyse du texte} :
Le contenu textuel du document doit être analysé sémantiquement de placer le document dans une ou plusieurs taxonomie. 
Ces informations seront ajoutées aux métadonnées du document.
\\
\item \textbf {transformation vectorielle du document} :
Le texte du document sera ensuite passé par un word2vec, qui transformera le contenu sémantique en vecteur de nombres flottants, qui sera utilisé pour retrouver les documents avec des contenus identiques.
Le document sera aussi passé dans un encodeur pour obtenir un vecteur permettant de retrouver les documents par ressemblance visuels par la suite.
\\
\item \textbf {Extraction d'informations utiles} :
Le document sera ensuite passé par une suite d'analyseurs pour en retirer des informations que le commanditaire aura jugé utiles.
Par exemple, on pourra trouver le nom des personnes mentionnées dans ce document, les dates de validité, les signatures, si le document est un faux, ... 
Ces informations seront ajoutées aux métadonnées du document.
\\
\item \textbf{Création d'une interface de recherche} :
Une interface de recherche sera ensuite réalisée afin de faciliter l'accès aux documents classés.
La méthode de recherche sera basée sur les informations présentées précédemment.
\\
\end {itemize}




