Notre équipe est composée de trois membres :
\begin {itemize}
\item \textbf {Baptiste Hudyma} :
Étudiant en 5 ème année, spécialisation systèmes embarqués.
SPOC, représentant de l'équipe, responsable de la partie analyseurs spécialisés. 

\item \textbf {Martin Olivier} : 
Étudiant en 5 ème année, spécialisation réalité virtuelle.
Spécialiste machine learning (auteur publié dans le domaine), responsable de la partie d'analyse sémantique.


\item \textbf {Staberlin Valentain} :
Étudiant en 5 ème année, spécialisation véhicules autonomes.
Responsable management de l'équipe, responsable de la partie moteur de recherche.
\\
\end {itemize}


Le projet a été découpé en trois parties principales :
\begin {itemize}
\item \textbf {Analyseurs spécifiques} : 
Les analyseurs spécifiques sont responsables de la récupération des informations dans un document, comme par exemple le nom des personnes concernées, ou les dates.
Ces données seront inclues dans les métadonnées de chaque documents.
\\
\item \textbf {Analyseurs sémantiques} :
L'analyseur sémantique se base sur le contenu textuel d'un document pour le classé avec les termes taxonomiques fournis par la prefecture.
Il est aussi responsable pour la production d'un word2vec.
\\
\item \textbf {Moteur de recherche} :
Le moteur de recherche est la dernière partie du projet, axée sur la réalisation d'un système et d'une interface de recherche se basant sur les métadonnées de chaque documents.
\\
\end {itemize}
