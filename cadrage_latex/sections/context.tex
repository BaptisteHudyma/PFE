
\subsection {Contexte}

Ce projet est proposé par la Prefecture de la Mayenne dans un contexte de dématérialisation des documents administratifs.
En effet dans un effort de numérisation de leur base documentaire, il devient complexe de gérer la grande quantité/diversité de documents informatisés. 
De plus, le facteur humain de la classification non automatisée n'est pas une garantie d'archivage pérenne.
En effet les employés administratifs changent, et d'une personne à l'autre la stratégie d'archivage peut et va varier. \\

C'est pourquoi la Prefecture de Mayenne envisage un projet de classification automatisée.
La mise en oeuvre de ce projet permettra de faciliter la recherche de documents, accélérer les démarches et réduire la masse documentaire de l'État (quantité de documents stockés), tout en garantissant que le système d'archivage garde une logique constante, indépendante du facteur humain.



\subsection {Analyse du problème}
%Plan : 
%1) la numérisation et le classement prend du temps
%2) c'est source d'erreurs car les personnes ne classent pas toutes de la même façon
%3) comme le classement n'est pas toujours correct, les recherches peuvent prendre du temps voire ne pas aboutir
%4) peut-être y-a-t-il des doublons???
%5) des documents périmés?....
%6) des documents incomplets ??? (toutes les pages pas présentes? ou bien les PJ rattachées incomplètes?)

%gain de temps des recherches
%securisation de l'information ?
%vérification de la complétude des dossiers


La Prefecture dispose d'un grand nombre de documents qui nécessitent d'être numérisés et classés par un système indépendant de l'intervention humaine, source d'erreur.
Ce travail laborieux est actuellement réalisé par des employés, ce qui rend cette tâche couteuse en temps et en ressources humaines.

C'est la la source d'erreur principal de ce classement : chaque employés utilisent des systèmes de classement différents, et n'emploient pas les mêmes pratiques d'archivage.
Cette différence dans les méthodes de classement se révèle être un problème lors de la recherche des documents archivés par la suite.
Ces recherches peuvent alors prendre beaucoup de temps, voir même ne pas aboutir.
\\
De plus certains documents peuvent se retrouver dupliqués, périmés ou incomplets, ce qui rend leur utilité moindre.
\\
\par
Ce projet permettra un gain de temps sur l'archivage et les recherches documentaires, et permettra une accélération durable des services préfectoraux.


\subsection {Périmètre}
Le type de documents a analyser ont été définis précisément lors de la réunion du vendredi 18 octobre 2019 sur site avec le commanditaire.
Les documents seront textuels, notamment des arrêtés préfectoraux, pouvant être en parti trouvés en ligne sur le site la préfecture.
Ces documents peuvent contenir des images ou des tableaux, ce qui va rendre difficile l'extraction de texte brut.
\\
Nous avons récupéré quelques documents types avec le commanditaire afin de commencer notre travail technique.
\\
En accord avec le commanditaire, nous avons décidé de limiter notre travail à des documents textuels en français.







